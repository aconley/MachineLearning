\font\big=cmbx10 scaled\magstep1
\font\tfont=cmr10 scaled\magstep3

\topglue 0.5in
\centerline{\tfont Notes on Elements of Statistical Learning}
\vskip 0.5in

\centerline{\big Chapter 4}
\vskip 0.2in

\noindent
{\bf Section 4.2}
In the discussion around equation (4.4), relating to linear classification,
it is claimed that it must  be the case that  $\sum_{k \in {\cal G}} {\hat f}_k\left(x\right) = 1$
as long as there is a column of ones in the {\bf X}.  This seems like
nonsense, but a little experimentation shows it is true -- why?

An important point is the the sum of each row in $\hat {\bf B}$ is zero
except for the first one, which sums to one.  The fact that the sum of
the $\hat f_k$ is unity follows from this immediately.  But why do 
the sums work the way they do?

Consider the formula for $\hat {\bf B}$:
$
\hat {\bf B} = \left({\bf X}^T {\bf X}\right)^{-1} {\bf X}^T {\bf Y}.
$
Next consider that each row of ${\bf Y}$ sums to one.
Then, the row sum of ${\bf B}$ satisfies
$\sum_j \hat{\bf B}_{ij} = \left({\bf X}^T {\bf X}\right)^{-1} {\bf X}^T {\bf 1}$.
This is just the equation for doing a multi-linear fit to a constant vector
of all 1s, where the obvious solution will have an intercept of 1 and
zero for all other coefficients.

\vskip 0.2in
\noindent
{\bf Section 4.5.1}
The peceptron algorithm minimizes a quantity
$$
D\left(\beta, \beta_0\right) = - \sum_{i \in {\cal M}} y_i \left(x_i^T \beta + \beta_0\right)
$$
which is only proportional to the distance to the hyperplane.  However,
the separating proportionality factor is $1 / ||\beta||$ (see eq.\ 4.40),
so as written this doesn't quite work.  One possibility is that the algorithm
requires $||\beta|| = 1$, since that defines the same separating plan; but
the update steps don't require this, so there seems to be a problem here.

\vskip 0.4in
\centerline{\big Chapter 5}
\vskip 0.2in
\noindent
{\bf Section 5.2} The mathermatical notation $_+$ isn't really explained, but
is common.  It is defined as
$$
 x_+ = \cases{ x, & if $x > 0$;\cr 0, & otherwise \cr }.
$$
Thus, something like $\left(x - \xi\right)_+$ is a unit slope line starting
at $x = \xi$ and 0 below that.

\vskip 0.2in
{\bf Section 5.4.1}
After eq.\ 5.15, it is stated that ${\bf H}_{\xi} {\bf H}_{\xi} = {\bf H}_{\xi}$.
The definition is ${\bf H}_{\xi} = {\bf B}_{\xi} \left( {\bf B}_{\xi}^T
 {\bf B}_{\xi}\right)^{-1} {\bf B}_{\xi}^T$.  So 
 ${\bf H}_{\xi} {\bf H}_{\xi} = {\bf B}_{\xi} \left({\bf B}_{\xi}^T {\bf B}_{\xi}
 \right)^{-1} {\bf B}_{\xi}^T {\bf B}_{\xi} \left({\bf B}_{\xi}^T {\bf B}_{\xi} 
 \right)^{-1} {\bf B}_{\xi}^T =
 {\bf B}_{\xi} \left({\bf B}_{\xi} {\bf B}_{\xi} \right)^{-1} \left({\bf B}_{\xi}^T
 {\bf B}_{\xi}\right) \left({\bf B}_{\xi} {\bf B}_{\xi} \right)^{-1} {\bf B}_{\xi}^T =
 {\bf H}_{\xi},$
 as claimed.  Note that this is only non-trivial because ${\bf B}_{\xi}$
 is not square -- if it were, then ${\bf H}_{\xi}$ would be the identity anyways.
 
 \vskip 0.2in
 {\bf Section 5.6}
Equation 5.30: $y \log p + \left(1 - y\right) \log \left(1-p\right) =
y \log e^f - y \log \left(1 + e^f\right) - \left(1 - y\right) \log \left(1 + e^f\right) =
y f - \log \left(1 + e^f\right)$, as claimed.

What about equation 5.31?  We have
$
{\partial \ell \over \partial \theta} = y {\partial f \over \partial \theta} - p 
{\partial f \over \partial \theta}$ minus a term for the derivative (see 5.11).
Then substitute ${\partial f \over \partial \theta} = {\bf N}$.
5.32 is basically the same thing.

\vskip 0.4in
\centerline{\big Chapter 7}
\vskip 0.2in
\noindent
{\bf Section 7.1} This is perhaps an obvious point, but the reason we would
prefer ${\rm Err}_{\cal T}$ to Err is that ultimately we always train with one
training set, and we would like to know the err for the one we actually use
than averaging over all possible training sets.

\vskip 0.2in
\noindent
{\bf Section 7.3} The bias variance decomposition takes some work to show,
but it's important.  The assumption is that $Y = f\left(x\right) + \epsilon$ with
$E\left(\epsilon\right) = 0$, and our estimate of $f$ is $\hat f$.  Dropping
the arguments from $f$ and ${\hat f}$ (so $f\left(x\right)$ is just $f$), at a point $x_0$,
$$
 \eqalign{
   E\left[ \left(Y - {\hat f}\right)^2 \right] 
    =& \, E\left[ \left(Y - f + f - {\hat f} \right)^2 \right] \cr
    =& \, E\left[ \left(Y - f\right)^2 \right] + 
            E\left[\left(f\ - {\hat f}\right)^2\right] +
         2 E\left[\left(f - {\hat f}\right)\,\left(Y - f\right)\right] \cr
    =& \, E\left[ \left(Y -f\right)^2 \right] + 
            E\left[\left(f - {\hat f}\right)^2\right] +
         2 \left( E\left[f {\hat f}\right] + E\left[Y f\right] 
                   -E \left[Y {\hat f}\right] - E\left[f^2\right]\right).
}
$$
Now, $Y - f = \epsilon$, so $E\left[\left(Y - f\right)^2\right] = E\left[\epsilon^2\right]$,
and since $E\left[\epsilon\right] = 0$, this is just ${\rm Var}\left[\epsilon\right]$.
Next, $E\left[y {\hat f}\right] = E\left[ f {\hat f}\right] + E\left[ \epsilon {\hat f}\right]$.
${\hat f}$ is deterministic and independent of $\epsilon$, so $E\left[\epsilon {\hat f}\right] = 0$.
Thus the first and third parts of the final term cancel.  Similarly, $E\left[ y f \right] = f^2$
and $E\left[ f^2\right] = f^2$ so the second and fourth also cancel.  Therefore,
$$
 E\left[ \left(Y - {\hat f}\right)^2 \right] = {\rm Var}\left[\epsilon\right] + 
 E\left[\left(f - {\hat f}\right)^2\right].
 $$
 
 Next, apply the same type of expansion to the final term but inserting $E\left[{\hat f}\right]$:
 $$
 \eqalign{
   E\left[ \left(f - {\hat f}\right)^2 \right] 
      =& \, E\left[ \left(f - E\left[{\hat f}\right] + E\left[{\hat f}\right] - {\hat f}\right)^2\right] \cr
      =& \, E\left[ \left(f - E\left[{\hat f}\right]\right)^2\right] + 
               E\left[\left(E\left[{\hat f}\right] - {\hat f}\right)^2\right]
              + 2 E\left[ \left(E\left[{\hat f}\right] - {\hat f}\right)\,\left(f - E\left[{\hat f}\right]\right)\right] .
 } 
 $$

As in the previous expansion, the final term turns out to cancel.  
$E\left[f E\left[{\hat f}\right]\right] = f E\left[{\hat f}\right]$ since $f$ is deterministic.
$E\left[ f {\hat f}\right] = f E\left[{\hat f}\right]$, so these terms cancel.
$E\left[E\left[{\hat f}\right]^2\right] = E\left[{\hat f}\right]^2$, and
$E\left[{\hat f} E\left[{\hat f}\right]\right] = E\left[{\hat f}\right]^2$, so those also cancel.

So, if we identify $E\left[ \left(f - E\left[{\hat f}\right]\right)^2\right]$ as 
${\rm Bias}^2\left[{\hat f}\right]$ and the second is
$E\left[\left(E\left[{\hat f}\right] - {\hat f}\right)^2\right] = E\left[{\hat f}^2\right] - E\left[{\hat f}\right]^2$,
which is ${\rm Var}\left[{\hat f}\right]$.  So, finally,
$$
E\left[ \left(Y - {\hat f}\right)^2 \right] = {\rm Var}\left[\epsilon\right]
 + {\rm Bias}^2\left[{\hat f}\right] + {\rm Var}\left[{\hat f}\right],
$$
which is the bias-variance decomposition.

\vskip 0.2in
\noindent
{\bf Section 7.10}
Note that the notation in eq.~7.48 is a little silly, if technically correct.  Obviously one
doesn't need to recompute the loss function $N$ times, but $K$ times.  It does show
you what to do with unequal partitions, however.

\end