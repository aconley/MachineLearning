\font\big=cmbx10 scaled\magstep1
\font\tfont=cmr10 scaled\magstep3

\topglue 0.5in
\centerline{\tfont Notes on BDA 3}
\vskip 0.5in

\noindent
{\big Chapter 1}
\nobreak
\vskip 0.2in

\noindent {\bf Section 1.3}
\nobreak
The formula before eq~1.1 is just the definition of conditional probability in the
continuous case; in the discrete case it takes the form $P\left(A | B\right) = 
P(A \cap B) / P(B)$.  The second step in eq.~1.4 is just this rule being
applied with $|y)$ along for the ride.

\noindent {\bf Section 1.4}
\nobreak
The hemophelia discussion also is implicitly assuming that a female with
two affected chromosomes won't survive long enough to have children.
This isn't a terrible assumption, although not entirely true.

If the third son is affected, the calculation becomes
$$
 Pr\left(\theta = 1 | y_1, y_2, y_3\right) = {0.5 \times 0.2 \over 0.5 \times 0.2 + 0 \times 0.8} = 1
$$

\noindent {\bf Section 1.9}
\nobreak
In the example for the inverse cdf, the exponential function is used.
That has $p \left(v\right) = \lambda \exp \left[ -\lambda v \right]$ for
$v > 0$.  The cdf is
$$
 F\left(v\right) = \int_0^v \, dx \lambda \exp \left[ -\lambda x \right] =
   \left[ \exp\left(-\lambda v\right) \right]_0^v = 1 - \exp\left(-\lambda v\right)
$$
and the inverse is
$$
  F^{-1}\left(v\right) = - \log \left(1 - v\right) / \lambda .
$$
\end