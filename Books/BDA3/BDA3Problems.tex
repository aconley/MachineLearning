\topglue 0.5in
\centerline{Problems from BDA 3}
\vskip 0.5in

\noindent
{\bf Chapter 1}
\vskip 0.2in

\noindent {\bf Problem 1} \hfil \break
\noindent a) $P\left(y\right) \sim 1/2 \left(N\left(\mu_1, \sigma^2\right) + 
N \left(\mu_2, \sigma^2\right)\right)$ \hfil \break
\noindent b) 
$$
\eqalign{
 P \left( \theta = 1 | y \right) &= {P\left(y | \theta = 1\right) P\left(\theta = 1\right)
  \over P\left(y | \theta = 1\right) P\left(\theta = 1\right) +
  P\left(y | \theta = 2\right) P\left(\theta = 2\right)} \cr &=
  \left(1 + \exp \left[ \left( 2 y \left(\mu_2 - \mu_1\right) + 
   \mu_2^2 - \mu_1^2\right) / 2 \sigma^2 \right] \right)^{-1} }.
$$

\noindent {\bf Problem 3}\hfil \break
\noindent We have $P\left(xx\right) = p^2$, $P\left(xX\right) = 2 p \left(1-p\right)$
(where we ignore ordering, so that Xx is counted as the same as xX),
and so $P\left(XX\right) = 1 - p^2 - 2 p \left(1-p\right) = \left(1 - p\right)^2$.
Now 
$$
P\left(xX | \rm{brown}\right) =  {1 \cdot P\left(xX\right) \over
1 \cdot P\left(xX\right) + 1 \cdot P\left(XX\right) + 0 \cdot P\left(xx\right)} = 
 {2 p \over 1 + p},
$$
and therefore $P\left(XX | \rm{brown}\right) = \left(1-p\right)/\left(1 + p\right)$.

The complication here is normalization.  So what we want to compute is
$$P\left(xX | p_1=\rm{brown}, p_2 = \rm{brown}\right) \over
P\left(xX | p_1=\rm{brown}, p_2 = \rm{brown}\right) +
P\left(XX | p_1=\rm{brown}, p_2 = \rm{brown}\right).$$
Starting with the first term
$$
\eqalign{
P\left(xX | p_1=\rm{brown}, p_2 = \rm{brown}\right)  &= P\left(xX | p_1=xX, p_2=xX\right)
 P\left(xX\right)^2 \cr &+ 2 P\left(xX | p_1=xX, p_2=XX\right) P\left(xX\right) P\left(XX\right)
 \cr &+ P\left(xX | p_1=xx, p_2=xx\right) P\left(xx\right)^2 \cr 
  &= {1 \over 2} \left({2 p \over 1+p}\right)^2 + 2 {1 \over 2} {2 p \over 1 + p} {1 - p \over 1 + p} + 0
  \left({1 - p \over 1 + p}\right)^2 \cr &= {2 p \over \left(1 + p\right)^2},
}
$$
where things like $P\left(xX | p_1=xX, p_2 = XX\right) = 1/2$ come from inheritance
squares.

Similarly
$$
\eqalign{
P\left(XX | p_1=\rm{brown}, p_2 = \rm{brown}\right)  &= P\left(XX | p_1=xX, p_2=xX\right)
 P\left(xX\right)^2 \cr &+ 2 P\left(XX | p_1=xX, p_2=XX\right) P\left(xX\right) P\left(XX\right)
 \cr &+ P\left(XX | p_1=xx, p_2=xx\right) P\left(xx\right)^2 \cr 
  &= {1 \over 4} \left({2 p \over 1+p}\right)^2 + 2 {1 \over 2} {2 p \over 1 + p} {1 - p \over 1 + p} + 
  1 \cdot \left({1 - p \over 1 + p}\right)^2 \cr &= {1 \over \left(1 + p\right)^2}.
}
$$
Therefore, the steady state fraction is
$${P\left(xX | p_1=\rm{brown}, p_2 = \rm{brown}\right) \over
P\left(xX | p_1=\rm{brown}, p_2 = \rm{brown}\right) +
P\left(XX | p_1=\rm{brown}, p_2 = \rm{brown}\right)} = {2 p \over 1 + 2 p}
$$
as claimed.

If Judy has brown eyed kids with a xX, the outcome depends on Judy's genome:
$$
\eqalign{
 P\left(\rm{child} = \rm{brown} | \rm{Judy}=xX\right) &= P\left(xX | J=xX\right)
 + P\left(XX | J=xX\right) = 1/2 + 1/4 = 3/4 \cr
 P\left(\rm{child} = \rm{brown} | \rm{Judy} = XX\right) &= 1}.
 $$
So, if Judy has $n$ brown eyed children, we have
$$
 P\left(J=xX | n\right) = {P\left(n | xX\right) P\left(xX\right) \over
  P\left(n | xX\right) P\left(xX\right) + P\left(n | XX\right) P\left(XX\right)} =
  {\left(3 \over 4\right)^n 2 p \over \left(3 / 4\right)^n 2 p + 1}
$$

\noindent {\bf Problem 6}\hfil \break
\noindent 
The proportion of births that are fraternal twins is 1/125, and 1/4 of those are
both males.  The proportion of identical twins is 1/300, and 1/2 of those are both
males.  Thus
$$
  P\left(i | \rm{both\,male}\right) = 
  {P\left( \rm{both\,male} | i\right) P\left(i\right) \over
   P\left( \rm{both\,male} | i\right) P\left(i\right) +
   P\left( \rm{both\,male} | f\right) P\left(f\right)} = 
    {1/4 \cdot 1/125 \over 1/4 \cdot 1/125 + 1 / 2 \cdot 1 / 300} = {6 \over 11}.
$$
\end